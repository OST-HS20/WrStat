\section{Wahrscheinlichkeit}
\subsection{Wahrscheinlichkeitsdichte}
\todo{Woche 6: Prüfungs aufgabe: Wahrscheinlichkeitsdichte muss 1 sein, Paramter anpassen.}
Die Wahrscheinlichkeitsdichte muss den Flächeninhalt $1$ haben.
\[
\int_{-\infty}^{\infty}\varphi(x)dx = 1
\]

\subsection{Verteilungsfunktion}
Die Verteilungsfunktion ist die Stammfunktion von der Wahrscheinlichkeitsdichte.
\todo{Woche 6}


\subsubsection{Gamma-Verteilung}
\todo{Woche 6}
\[
\gamma_{\alpha,v}(x) = \frac{1}{\Gamma(v)\alpha^vx^{v-1}e^{\alpha x}}
\]

\subsubsection{Normalverteilung}
\todo{Woche 8}
Summe vieler kleiner Einflüsse wie Messwerte oder wiederholte Experimente. Zwei Drittel aller Werte liegen innerhalb einer Standardabweichung $\sigma$ um den Erwartungswert $\mu$. Diese Verteilung wird verwendet, wenn $P$ irgendwo in der mitte liegt, für seltene (kleine $P$) kann die Piossonverteilung verwendet werden.
\[
\varphi(x, \mu) = \frac{1}{\sqrt{2\pi}\sigma}e^{-\frac{(x-\mu)^2}{2\sigma^2}}
\]

\subsubsection{Rechenregeln}
\textbf{Linearität} und \textbf{Summensatz}
\todo{Beispiel von Woche 8 20:30min, oder Tabelle}

\subsubsection{Exponentialverteilung}
\todo{Woche 7}
Gedächnislose Prozesse wie Radioaktiver Zerfall oder Warteschlangen.
\[\varphi(x) = \begin{cases*}
	ae^{-ax} \qquad x \geq 0 \\
	0 \qquad \text{sonst}
\end{cases*}\]
Parameter:$a. \frac{1}{a}$ = Mean Time between Failure

\textbf{Halbwertszeit}
Die Halbwertszeit ist auch der Median welcher durch 
\[
\med T = \frac{1}{a}\ln2
\]
bestimmt ist.

\subsubsection{Poissonverteilung}
Gedächnislose Prozesse $T_i$ mit gleichem $a$. Wobei $k$ die Anzahl der Prozesse und $\lambda = al$ mit $l$ der Länge auf dem Intervall ist. $\lambda$ kann auch eine Rate sein. Dies sollte nur verwendet werden, wenn $n$ gross und $P$ selten ist.
\[
P_k(\lambda) = e^{-\lambda}\frac{\lambda^k}{k!}
\]


\subsection{Binomialverteilung}
\todo{Woche 9}
\[
P(X = k) = \begin{pmatrix}	n \\ k \end{pmatrix} p^k(1-p)^{n-k}
\]

\subsection{Hypergeometrische Verteilung}
Von $N$ Objekten sind $M$ markiert. Daraus werden $n$ Objekte zufällig ausgewählt.
\[
P(X=m) = \frac{\begin{pmatrix}	M \\ m\end{pmatrix}\begin{pmatrix}	N-M \\ n-m\end{pmatrix}}{\begin{pmatrix} N \\ N\end{pmatrix}}
\]

\subsection{$\chi^2$-Verteilung}
Siehe \ref{chi-verteilung}